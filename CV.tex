\documentclass[11pt,a4paper]{moderncv}
\moderncvstyle{classic}
\moderncvcolor{blue}
\usepackage[scale=0.8]{geometry}
\usepackage{multicol}
\usepackage{academicons}
\usepackage{lmodern}


% Personal Information
\name{Benaissa}{Dekhici}
\title{Researcher in Bioenergy \& Data-Driven Innovations}
\extrainfo{
    \faEnvelope\enspace\href{mailto:b.dekhici@surrey.ac.uk}{b.dekhici@surrey.ac.uk} \quad 
    \faMobile\enspace+44 7414 294968 \quad
    \faGlobe\enspace\href{https://benmola.github.io}{Webpage} \\
    \faGraduationCap\enspace\href{https://scholar.google.com/citations?user=Benaissa+Dekhici}{Google Scholar} \quad
    \faResearchgate\enspace\href{https://www.researchgate.net/profile/Benaissa-Dekhici}{ResearchGate} \quad
    \faLinkedin\enspace\href{https://www.linkedin.com/in/Benaissa-Dekhici}{LinkedIn} \quad
    \faGithub\enspace\href{https://github.com/benmola}{GitHub}
}


\begin{document}

\makecvtitle

\section{About Me}
\cvitem{}{\textit{I am a researcher in Bioenergy and Data-Driven Innovations, with a focus on the intersection of engineering, data science, and sustainable energy systems. My work involves developing advanced modeling, control, and data analytics solutions to address real-world challenges in bioenergy and environmental engineering. I am passionate about leveraging technology to drive progress in sustainability and create a positive impact on a global scale. Extensive travel experience having lived in Algeria, Italy, Turkey, and the UK, providing multicultural perspective and adaptability.}}

\section{Work Experience}
\cventry{Feb 2024--Present}{PDRA in Bioenergy Process Optimisation and Control}{University of Surrey}{England, UK}{}{\begin{itemize}\item Focusing on advanced optimisation under uncertainty for bioenergy industry digitalization\item Linked to UKRI Supergen Bioenergy Impact Hub\item Developing cutting-edge control strategies for sustainable energy systems\end{itemize}}
\cventry{Sept 2020--Jun 2021}{Teaching Assistant}{University of Tlemcen}{Tlemcen, Algeria}{}{\begin{itemize}\item Courses: Linear Multivariable Systems, Nonlinear Systems, Optimal Control\item Supervised undergraduate and graduate students in advanced control theory\end{itemize}}
\cventry{Sept 2019--Jun 2020}{Teaching Assistant}{University of Tlemcen}{Tlemcen, Algeria}{}{\begin{itemize}\item Courses: Multivariable Systems, Nonlinear Systems\item Developed practical laboratory exercises and assessment materials\end{itemize}}
\cventry{Jun 2019--Sept 2019}{Trainer/Teacher}{FROMAC Academy}{Tlemcen, Algeria}{}{\begin{itemize}\item Subject: Automatics and Industrial Data Processing\item Delivered professional training programs to industry professionals\end{itemize}}
\cventry{May 2017--Jun 2019}{Research Support State Engineer}{Research Center in Industrial Technologies, CRTI}{Algiers, Algeria}{}{\begin{itemize}\item Responsible for drone systems development (hardware and software)\item Led interdisciplinary teams in UAV technology advancement.\end{itemize}}
\cventry{Sept 2016--Oct 2016}{Trainee as Automation Engineer}{LATAFNA Mill}{Tlemcen, Algeria}{}{\begin{itemize}\item Gained hands-on experience in industrial automation systems\item Worked on process control and optimization projects\end{itemize}}
\cventry{Since 2018}{Researcher}{Tlemcen Automatics Laboratory LAT}{Tlemcen, Algeria}{}{\begin{itemize}\item Active member contributing to laboratory research initiatives.\end{itemize}}

\section{Education}
\cventry{2018--2024}{Ph.D. in Automatics}{University of Tlemcen}{Tlemcen, Algeria}{}{
\textbf{Thesis:} ``Data-Driven Modeling, Order Reduction and Control of Anaerobic Digestion Processes''\\
\textbf{Supervisors:} Prof. Boumediene Benyahia \& Prof. Brahim Cherki\\
\textbf{Co-direction:} LBE-INRAE Narbonne, France\\
\textbf{International Mobility:}
\begin{itemize}
\item Bilateral Student at University of Trento (Aug 2022--Jul 2023)
\item International Credit Mobility Student at University of Trento (Aug 2021--Jul 2022)
\end{itemize}}
\cventry{2013--2015}{M.Sc. in Automatics and Industrial Data Processing}{University of Tlemcen}{Tlemcen, Algeria}{}{}
% \textbf{Thesis:} ``Commande d'un Quadrirotor Parrot Bebop Drone''\\
% \textbf{Supervisors:} Prof. Brahim Cherki \& Dr. Mokhtari Mohammed Rida.}
\cventry{2009--2013}{B.Sc. in Automatics}{University of Tlemcen}{Tlemcen, Algeria}{}{}

% \section{Publications}
% \cvitem{2024}{\textit{A data-driven Koopman-based approach for the identification of switched nonlinear systems}. International Journal of Systems Science, 55(4), 503-518.}
% \cvitem{2023}{\textit{Data-driven reduced-order modeling of anaerobic digestion process using Koopman operator}. Processes, 11(11), 3071.}
% \cvitem{2021}{\textit{A new UAV design based on the Coanda effect for inspection and monitoring missions}. In 2021 6th International Conference on Control, Automation and Diagnosis (ICCAD), (pp. 1-6). IEEE.}
% \cvitem{2020}{\textit{A new design of a UAV based on Coanda effect for inspection and monitoring missions}. In CARI 2020-15th African Conference on Research in Computer Science and Applied Mathematics.}

\section{Research Projects}
\cventry{Feb 2024--Present}{Rapid Digitalisation of Bioenergy for Higher Efficiency and Profit}{UKRI Supergen Bioenergy Impact Hub}{}{}{Developing advanced optimization frameworks to transform the bioenergy industry into a data-driven, digitalized Industry.}
\cventry{Jan 2025--Aug 2025}{Biomethane Islands – Feasibility Study}{Future Energy Networks: Network Innovation Allowance}{}{}{Developed an optimisation model for designing anaerobic digestion sites for Biomethane Islands, incorporating product demands, feedstock supply, environmental factors, and sizing/costing.}
\cventry{Nov 2024--Feb 2025}{D-Xpert: AI-Based Recommender System for Smart Energy Saving}{Innovate UK Project}{}{}{Dynamic Heat Flow Model Development, HVAC Profiling, AI Occupancy Model, and Model Predictive Control Algorithm development.}
\cventry{Jul 2024--Dec 2024}{Integrating CFD Modeling and Kinetics for Enhanced Anaerobic Digestion}{The Carbon Recycling Network Business Interaction Voucher}{}{}{Developed automated methodology integrating CFD with kinetic models and Bayesian Optimisation for optimizing anaerobic digester mixing systems.}
\cventry{Oct 2024}{Techno-economic Analysis of Novel Water Treatment System}{Consultancy with Intelligent Tomorrow Ltd}{}{}{Developed base simulation for mass and energy balance, cost estimation, designed pilot system, and delivered profitability assessment.}

\section{Technical Skills}
\cvitem{Process Engineering}{Bioenergy Systems, Process Systems Engineering, Anaerobic Digestion Processes, AD and Biogas Expert}
\cvitem{Data Science}{Machine Learning, Dynamic System Identification, Data-Driven Approaches, Model Order Reduction, Control Systems, Industrial Informatics, Artificial Intelligence}
\cvitem{Programming}{Python (Advanced), MATLAB/Simulink (Advanced), C++ (Intermediate)}
\cvitem{Languages}{English (Fluent), French (Fluent), Arabic (Native)}

% \section{Activities}
% \cvitem{2024}{Participation in the General Assembly of the SUPERGEN Bioenergy Hub, Manchester, UK.}
% \cvitem{2023}{Participation in the 1st ADM1 International Symposium, Milan, Italy.}
% \cvitem{2022}{Participation in the 17th International Conference on Research in Computer Science and Applied Mathematics (CARI), Yaoundé, Cameroon.}
% \cvitem{2021}{Participation in IEEE International Conference on Control, Automation and Diagnosis (ICCAD), Grenoble, France.}
% \cvitem{2020}{Participation in the 15th African Conference on Research in Computer Science and Applied Mathematics (CARI), Thiès, Senegal.}

\section{Hobbies \& Interests}
\cvitem{Research: }{Reading research articles, ML tools exploration, science books, chess}
\cvitem{Gaming: }{Playing and watching football, video games across all consoles, triple-A games, Nintendo Switch gaming}
\cvitem{Technology:}{Tech enthusiast (IT, electronics), DIY projects and electronics, 3D printing, Electronic chips and boards, Operating systems exploration}
%\cvitem{Travel:}{Extensive travel experience having lived in Algeria, Italy, Turkey, and the UK, providing multicultural perspective and adaptability}
\end{document}
